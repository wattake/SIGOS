\section{議論}
\label{sec:disc}

\subsection{エラー伝搬}

\sysname は Unikernel コンポーネントの高い強度での分離を
実現する CubicleOS~\cite{SartakovEtAl-ASPLOS21}や FlexOS~\cite{LefeuvreEtAl-FlexOS}と同じ強度で分離を実現するものであるものの,
このメカニズムで防ぐことができないエラー伝搬が存在する.
\sysname の分離メカニズムによって防ぐことができないエラー伝搬は,
関数呼び出し先のコンポーネントのエラーが関数呼び出し元のコンポーネントへと伝搬するケースである.
このケースは,正当な制御フローの範囲内にバグが存在するときに発生するものであり,
関数実行中のコンポーネントが呼び出し元のコンポーネントのデータを不当に更新することである.

このケースのエラー伝搬を完全に防ぐためには,
他のコンポーネントのデータ更新時の正当性を関数ごとに検証する必要がある.
しかし,Unikernel はアプリケーションの要件によって,コンポーネントの構成が異なるため,
同じ関数であっても,関数の呼び出し元と呼び出し先のコンポーネントの関係が一定であるとは限らず,
データ更新の正当性を定義することが難しい.


